\documentclass[a4paper, 12pt]{article}
\usepackage[total={17cm,25cm}, top=2.5cm, left=2.5cm, right=2.5cm,  includefoot]{geometry}
\usepackage[utf8]{inputenc}
\usepackage{array}
\usepackage{multirow}
\usepackage{hhline}
\usepackage{gensymb}
\usepackage{graphicx}
\graphicspath{ {} }
\usepackage[czech]{babel}
\usepackage{enumitem}
\usepackage{pdfpages}
\usepackage{amsmath}
\usepackage{verbatim}
\usepackage{listings}
\usepackage{hyperref}
\usepackage{amssymb}


\pagestyle{empty} % vypne číslování stránek




%\usepackage[OT2,OT1]{fontenc}
\newcommand\cyr
{
\renewcommand\rmdefault{wncyr}
\renewcommand\sfdefault{wncyss}
\renewcommand\encodingdefault{OT2}
\normalfont
\selectfont
}
\DeclareTextFontCommand{\textcyr}{\cyr}
\def\cprime{\char"7E }
\def\cdprime{\char"7F }
\def\eoborotnoye{\char’013}
\def\Eoborotnoye{\char’003}


\begin{document}



\begin{titlepage}
\begin{center}
\noindent
\Large \textbf{České vysoké učení technické v Praze }\\ Fakulta stavební
\vspace{5cm}

\huge

%vložení loga cvut
%\begin{figure}[h!]
%	\centering
%	\includegraphics[width=7cm]{logo.png}
%\end{figure}

\vspace{0.5cm}

155ADKG: Digitální model terénu \\

\vspace{10cm}




\Large
Michael Kala\\
Anna Zemánková \\

\end{center}

\end{titlepage}




\pagestyle{plain}     % zapne obyčejné číslování
\setcounter{page}{1}  % nastaví čítač stránek znovu od jedné

%\tableofcontents
%\newpage

\section{Zadání}

\begin{figure}[h!]
	\includegraphics[clip, trim=0cm 5cm 0cm 3cm, width=1.0\textwidth]{zadani.pdf}
\end{figure}


\section{Údaje o bonusových úlohách}




\clearpage

\section{Popis a rozbor problému}

Mějme množinu bodů $P \{p_i\}$,  $p_i = \{x_i, y_i, z_i\}$. Nad touto množinou chceme vytvořit síť trojúhelníků $t_j$ pomocí Delaunay triangulace $DT$, následně vytvořit vrstevnice a pro vizualizaci DMT určit sklon a expozici jednotlivých trojúhelníků.\\ 

\subsection{Delaunay triangulace}
Vlastnosti:
\begin{itemize}
\item Uvnitř kružnice opsané trojúhelníku $t_j \in DT$ neleží žádný jiný bod množiny P.
\item $DT$ maximalizuje minimální úhel v $\forall t_j$, avšak $DT$ neminimalizuje maximální úhel v $t_j$.
\item $DT$ je lokálně optimální i globálně optimální vůči kritériu minimálního úhlu.
\item $DT$ je jednoznačná, pokud žádné čtyři body neleží na kružnici.
\end{itemize}
\vspace{1.5cm}

\subsection{Vrstevnice}
Vrstevnice byly určeny za využití lineární interpolace, při které se předpokládá, že spád terénu je mezi podrobnými body $p_i$, mezi nimiž se provádí interpolace, konstantní.\\

Mějme trojúhelník $t_j$, tvořený hranami $e_1, e_2, e_3$ a rovinu vrstevnice $\rho$ o dané výšce.

Vztah hrany tojúhelníku a roviny vrstevnice:
\begin{enumerate}
\item $(z-z_i)*(z-z_{i+1}) < 0$  $\longrightarrow e_i \cap \rho$
\item $(z-z_i)*(z-z_{i+1}) > 0$  $\longrightarrow e_i \notin \rho$ 
\item $(z-z_i)*(z-z_{i+1}) = 0$  $\longrightarrow e_i \in \rho$
\end{enumerate}

\noindent Pokud  $e_1, e_2, e_3 \in \rho$, jedná se o trojúhelník náležící rovině $\rho$ a není nutné vrstevnici pro tento trojúhelník řešit.\\
Jestliže $e_i \cap \rho$, je vypočten průsečík hrany $e_i = (p_1, p_2)$ a roviny vrstevnice $\rho$ o výšce $z$:
$$ x = \frac{(x_2-x_1)}{(z_2-z_1)}(z-z_1)+x_1, $$
$$ y = \frac{(y_2-y_1)}{(z_2-z_1)}(z-z_1)+y_1.$$
\clearpage

\subsection{Sklon}

Sklon je úhel $\varphi$ mezi svislicí $n$ a normálou trojúhelníku $n_t$. Rovina trojúhelníku $t_j$ je určena vektory u,v.\\

\noindent$$n = (0,0,1)$$
$$n_t = \vec{u}\times \vec{v}$$
$$\varphi =\arccos(\frac{n_t  n}{|n_t| |n|})$$



\subsection{Expozice}
Expozice je orientace trojúhelníku vůči světovým stranám.\\
$$A = \arctan2(\frac{n_x}{n_y});$$ kde $n_x, n_y$ jsou vektorové součiny $u$ a $v$.



\clearpage
\section{Popis algoritmů}

\subsection{Delaunayova triangulace}
Triangulace byla realizována metodou inkrementální konstrukce, body jsou tedy do triangulace přidávány postupně a to tak, aby vybraný bod ležel v levé polorovině od orientované hrany, poloměr opsané kružnice byl minimální a zároveň jsou preferovány body, jejichž střed opsané kružnice leží v pravé polorovině. Pokud žádný bod těmto kritériím nevyhovuje, je orientace hrany obrácena a bod je vybírán znovu. Jakmile je bod nalezen, jsou k němu vytvořeny orientované hrany a vše je uloženo do triangulace.\\
Pro "manipulaci" s hranami se používá struktura Active Edges List $AEL$, do ní jsou ukládány hrany, ke kterým je třeba nalézt třetí bod a vytvořit trojúhelník. Jakmile je $AEL$ prázdná, algoritmus končí.

\subsubsection{Implementace metody}
\begin{enumerate}
\item $ p_1 = rand(P), ||p_2-p_1|| = min $ ....náhodný a nebližší bod
\item Vytvoř hranu $ e = (p_1,p_2) $ 
\item Inicializuj: $p_{min} = arg min_{\forall p_i\in\sigma_L(e)} r'(k_i), k_i = (a, b, p_i), e = (a,b)$
\item Pokud $ \nexists p_{min},$ prohoď orientaci $e \longleftarrow (b,a) $. Jdi na $3)$
\item $e_2 = (p_1,p_{min}), e_3 = (p_{min},p_1) $...zbývající hrany trojúhelníku
\item $AEL \longleftarrow e, AEL \longleftarrow e_2, AEL \longleftarrow e_3 $
\item $DT \longleftarrow e, DT \longleftarrow e_2, DT \longleftarrow e_3 $ 
\item while AEL not empty:
\item \hspace {1cm} $AEL  \longrightarrow e, e = (p_1, p_2) $....vezme první hranu z AEL
\item \hspace {1cm}$ e = (p_2, p_1)$ ...prohodí její orientaci
\item \hspace {1cm} $p_{min} = arg min_{\forall p_i\in\sigma_L(e)} r'(k_i), k_i = (a, b, p_i), e = (a,b) $
\item \hspace {1cm} if $ \exists p_{min}:$
\item \hspace {2cm} $e_2 = (p_1,p_{min}), e_2 = (p_{min},p_1) $...zbývající hrany trojúhelníku
\item \hspace {2cm} $DT \longleftarrow e  $  
\item \hspace {2cm} $ add(e_2,AEL,DT), add(e_3,AEL,DT)$
\end{enumerate}
\clearpage
Dílčí algoritmus Add:
\begin{enumerate}
	\item Vytvoř hranu $e' = (b,a)$
	\item if $(e' \in AEL)$
	\item \hspace {1cm} $ AEL \longrightarrow e'$...Odstraň z AEL
	\item else:
	\item \hspace {1cm} $ AEL \longleftarrow e$ ...Přidej do AEL
	\item $DT\longleftarrow (a,b)$ Přidej do DT
\end{enumerate}

\subsubsection{Problematické situace}
Pokud je vstupními daty grid, dochází k nejednoznačnosti Delaunay triangulace - na opsané kružnici leží čtyři body. V tom případě nefunguje výpočet vrstevnic ani následných analýz.

%OBRÁZEK
%\begin{figure}[h]
%	\centering
%	\includegraphics[width=10cm]{grid.jpg}
%	\caption{Delaunay triangulace - grid}
%\end{figure}

%OBRÁZEK
%\begin{figure}[h]
%	\centering
%	\includegraphics[width=10cm]{vrstevnice_grid.jpg}
%	\caption{Vrstevnice - grid}
%\end{figure}

%OBRÁZEK
%\begin{figure}[h]
%	\centering
%	\includegraphics[width=10cm]{sklon_grid.jpg}
%	\caption{Sklon - grid}
%\end{figure}
\clearpage

% ===============================================================

\section{Vstupní data}

Vstupními daty je množina bodů, kterou lze zadat dvěma způsoby:\\
\begin{enumerate}
\item klikáním na kanvas, výška se generuje náhodně
\item nahráním textového souboru se souřadnicemi bodů [X,Y,Z] *.txt pomocí tlačítka \textit{Load points}
\end{enumerate}
%--------------------------------------------------------------------------------------------------------------------------------------
Součástí příloh je textový soubor s testovacími daty \textit{testovaci data.txt}.

\section{Výstupní data}
Výsledky aplikace jsou vizualizovány v kanvasu grafického rozhraní.

%OBRÁZEK
%\begin{figure}[h]
%	\centering
%	\includegraphics[width=10cm]{sklon.jpg}
%	\caption{Sklon}
%\end{figure}

\clearpage

\section{Ukázka vytvořené aplikace}


%OBRÁZEK
%\begin{figure}[h]
%	\centering
%	\includegraphics[width=10cm]{aplikace.jpg}
%	\caption{Aplikace}
%\end{figure}

Nejprve je třeba zadat vstupní data - viz kapitolu Vstupní data a následně je možné vypočítat Delaunay triangulaci, teprve poté jsou zpřístupněna i ostatní tlačítka aplikace. \\

%OBRÁZEK
%\begin{figure}[h]
%	\centering
%	\includegraphics[width=10cm]{DT.jpg}
%	\caption{Delaunay triangulace}
%\end{figure}

Při výpočtu vrstevnic je třeba zadat krok vrstevnic a (ne)zaškrtnout zvýraznění hlavních vrstevnic.

%OBRÁZEK
%\begin{figure}[h]
%	\centering
%	\includegraphics[width=10cm]{vrstevnice.jpg}
%	\caption{Vrstevnice}
%\end{figure}

V sekci analysis je možné vypočíst a vizualizovat sklon a expozici.
%OBRÁZEK
%\begin{figure}[h]
%	\centering
%	\includegraphics[width=10cm]{sklon.jpg}
%	\caption{Sklon}
%\end{figure}

%OBRÁZEK
%\begin{figure}[h]
%	\centering
%	\includegraphics[width=10cm]{expozice.jpg}
%	\caption{Expozice}
%\end{figure}

Vše je uvedeno do původního stavu (smazány body i vypočtené výsledky) kliknutím na tlačítko Clear.\\
\clearpage





\clearpage

%=======================================================================================



\subsection{Algorithms}
V třídě Algorithms jsou staticky implementovány algoritmy počítající kovexní obálku a minimální ohraničující obdélník (včetně vodící linie).

\begin{itemize}

	\item Výčtový typ \textbf{TPosition}
		\begin{itemize}
			\item Typ využitý jako návratová hodnota členské metody \textbf{getPointLinePosition}.
			\item \textbf{LEFT = 0}
			\item \textbf{RIGHT = 1}
			\item \textbf{ON = 2}
		\end{itemize}

	\item Metoda \textbf{getPointLinePosition}
		\begin{itemize}
			\item Tato metoda slouží k určení polohy bodu vůči přímce. Návratovou hodnotou je výčtový typ \textbf{TPosition}.
			\item Vstup
				\begin{itemize}
					\item \textbf{QPointF \&q} - určovaný bod
					\item \textbf{QPointF \&a, \&b} - body přímky
				\end{itemize}
			\item Výstup
				\begin{itemize}
					\item \textbf{LEFT} - bod vlevo od přímky
					\item \textbf{RIGHT} - bod vpravo od přímky
					\item \textbf{ON} - bod na přímce
				\end{itemize}

		\end{itemize}

	\item Metoda \textbf{getTwoVectorsAngle}
		\begin{itemize}
			\item Tato metoda slouží k určení úhlu mezi 2 přímkami. Její návratovou hodnotu je \textbf{double}.
			\item Vstup
				\begin{itemize}
					\item \textbf{QPointF \&p1, \&p2} - body první přímky
					\item \textbf{QPointF \&p3, \&p4} - body druhé přímky
				\end{itemize}
		
			\item Výstup
				\begin{itemize}
					\item Úhel mezi 2 přímkami
				\end{itemize}			
		\end{itemize}

	\item Metoda \textbf{getPointLineDistance}
		\begin{itemize}
			\item Tato metoda slouží k výpočtu vzdálenosti bodu od přímky. Její návratovou hodnotou je \textbf {double} %\begin{double.}
			\item Vstup
				\begin{itemize}
					\item \textbf{QPointF \&q} - určovaný bod
					\item \textbf{QPointF \&a, \&b} - body přímky
				\end{itemize}
			\item Výstup
				\begin{itemize}
					\item Vzdálenost bodu od přímky
				\end{itemize}
		\end{itemize}

	\item Přetížená metoda \textbf{rotateByAngle}
		\begin{itemize}
			\item Tato metoda slouží k rotaci dané množiny o úhel. Jejím návratovým typem je \textbf{void}.
			\item Vstup
				\begin{itemize}
					\item Přetížení 1
						\begin{itemize}
							\item \textbf{std::vector$<$QPointF$>$ \&points} - vektor bodů, jež má být orotován
							\item \textbf{double angle} - úhel, o který má rotace být provedena
 						\end{itemize}
					
					\item Přetížení 2
						\begin{itemize}
							\item \textbf{QPolygonF \&points} - polygon, jež má být orotován
							\item \textbf{double angle} - úhel, o který má rotace být provedena
 						\end{itemize}

					\item Přetížení 3
						\begin{itemize}
							\item \textbf{QLineF \&points} - úsečka, jež má být orotována
							\item \textbf{double angle} - úhel, o který má rotace být provedena
 						\end{itemize}					
				\end{itemize}
			
		\end{itemize}

	\item Metoda \textbf{getDistance}
		\begin{itemize}
			\item Tato metoda slouží k výpočtu vzdálenosti dvou bodů. Jejím výstupním typem je \textbf{double}.
			\item Vstup
				\begin{itemize}
					\item \textbf{QPointF \&a, \&b} - body, mezi kterými je vzdálenost počítána
				\end{itemize}
			\item Výstup
				\begin{itemize}	
					\item Vypočtená vzdálenost
				\end{itemize}
		\end{itemize}

	\item Metoda \textbf{jarvisScanCH}
		\begin{itemize}
			\item Tato metoda slouží k výpočtu konvexní obálky pomocí algoritmu Jarvis Scan. Během výpočtu je ošetřována singularita existence kolineárních bodů v datasetu. Jejím výstupním typem je \textbf{QPolygonF}.
			\item Vstup
				\begin{itemize}
					\item \textbf{std::vector} $<$\textbf{QPointF}$>$ \textbf{points} - vektor bodů, kolem nichž má být vytvořená konvexní obálka.
				\end{itemize}
			\item Výstup
				\begin{itemize}
					\item Polygon obsahující kovexní obálku.
				\end{itemize} 
		\end{itemize}

	\item Metoda \textbf{grahamScanCH}
		\begin{itemize}
			\item Tato metoda slouží k výpočtu konvexní obálky pomocí algoritmu Graham Scan. Jejím výstupním typem je \textbf{QPolygonF}.
			\item Vstup
				\begin{itemize}
					\item \textbf{std::vector} $<$\textbf{QPointF}$>$ \textbf{points} - vektor bodů, kolem nichž má být vytvořená konvexní obálka.
				\end{itemize}
			\item Výstup
				\begin{itemize}
					\item Polygon obsahující kovexní obálku.
				\end{itemize} 
		\end{itemize}

	\item Metoda \textbf{quickHullCH}
		\begin{itemize}
			\item Tato metoda slouží k výpočtu konvexní obálky pomocí algoritmu Quick Hull. Jejím výstupním typem je \textbf{QPolygonF}.
			\item Vstup
				\begin{itemize}
					\item \textbf{std::vector} $<$\textbf{QPointF}$>$ \textbf{points} - vektor bodů, kolem nichž má být vytvořená konvexní obálka.
				\end{itemize}
			\item Výstup
				\begin{itemize}
					\item Polygon obsahující kovexní obálku.
				\end{itemize} 
		\end{itemize}

	\item Metoda \textbf{quickHullLocal}
		\begin{itemize}
			\item Pomocná metoda k výpočtu konvexní obálky metodou Quick Hull. Jejím výstupním typem je \textbf{void}.
			\item Vstup
				\begin{itemize}
					\item \textbf{int s, e} - index počátečního a koncového bodu dělící přímky
					\item \textbf{std::vector} $<$\textbf{QPointF}$>$ \&\textbf{points} - vektor bodů, kolem nichž má být vytvořená konvexní obálka.
					\item \textbf{QPolygonF \&poly\_ch} - polygon obsahující body konvexní obálky
				\end{itemize}
			\item Výstup
				\begin{itemize}
					\item Polygon obsahující kovexní obálku.
				\end{itemize} 
		\end{itemize}

	\item Metoda \textbf{sweepLineCH}
		\begin{itemize}
			\item Tato metoda slouží k výpočtu konvexní obálky pomocí algoritmu Sweep Line. Jejím výstupním typem je \textbf{QPolygonF}.
			\item Vstup
				\begin{itemize}
					\item \textbf{std::vector} $<$\textbf{QPointF}$>$ \textbf{points} - vektor bodů, kolem nichž má být vytvořená konvexní obálka.
				\end{itemize}
			\item Výstup
				\begin{itemize}
					\item Polygon obsahující kovexní obálku.
				\end{itemize} 
		\end{itemize}

	\item Metoda \textbf{generatePoints}
		\begin{itemize}
			\item Metoda pro generování zadaného počtu a tvaru bodů. Jejím výstupním typem je \textbf{std::vector} $<$\textbf{QPointF}$>$ \textbf{points}.
			\item Vstup
				\begin{itemize}
					\item \textbf{QSizeF \&canvas\_size} - rozměry kreslícího plátna, ze kterých se determinuje rozsah generovaných bodů
					\item \textbf{int point\_count} - počet bodů, který se má generovat
					\item \textbf{std::string shape} - tvar vytvářené množiny bodů (random, grid, na kružnici, na elipse, na čtverci)
				\end{itemize}
			\item Výstup
				\begin{itemize}
					\item Vektor nagenerovaných bodů.
				\end{itemize}
		\end{itemize}

	\item Metoda \textbf{minimalRectangle}
		\begin{itemize}
			\item Metoda pro výpočet minimálního ohraničujícího obdélníku a hlavní linie. Jejím výstupním typem je \textbf{void}.
			\item Vstup
				\begin{itemize}
					\item \textbf{QPolygonF \&poly\_ch} - polygon obsahující konvexní obálku
					\item \textbf{QPolygonF \&minimal\_rectangle} - polygon, do kterého jsou počítány body minimálního ohraničujícího obdélníku
					\item \textbf{QLineF \&direction} - hlavní linie minimálního ohraničujícího obdelníku (resp. do této proměnné je počítaná)
					\item \textbf{bool compute\_dir\_line} - ukazatel určující zda-li má být počítána hlavní linie minimálního ohraničujícího obdélníku
				\end{itemize}

		\end{itemize}
\end{itemize}
\clearpage

\subsection{Draw}
Třída draw slouží k vykreslení vygenerovaných (nebo naklikaných) bodů, vypočteného minimálního ohraničujícího obdélníku a hlavní linie minimálního ohraničujícího obdélníku. V této třídě jsou zároveň nagenerované body zbavené duplicit a vypočtené konvexní obálky se zde omezují na striktní konvexní obálky (vše v metodě \textbf{setCH}. Třída dědí od třídy \textbf{QWidget}. 

\begin{itemize}
	\item Členské proměnné
		\begin{itemize}
			\item \textbf{std::vector} $<$\textbf{QPointF}$>$ \textbf{points} - vektor obsahující nagenerované nebo naklikané body
			\item \textbf{QPOlygonF ch} - polygon obsahující body konvexní obálky
			\item \textbf{QPolygonF rect} - polygon obsahující body minimálního ohraničujícího obdélníku
			\item \textbf{QLineF direction} - hlavní linie minimálního ohraničujícího obdélníka
		\end{itemize}

	\item Metoda \textbf{paintEvent}
		\begin{itemize}
			\item Tato metoda slouží k vykreslení nagenerovaných (nebo naklikaných) bodů, konvexní obýlky, minimálního ohraničujícího obdélníka a hlavní linie minimálního ohraničujícího obdélníka. Metoda se volá pomocí metody \textbf{repaint()}. Návratovým typem je \textbf{void}.
			\item Vstup
				\begin{itemize}
					\item \textbf{QPaintEvent *e}
				\end{itemize}
		\end{itemize}

	\item Metoda \textbf{mousePressEvent}
		\begin{itemize}
			\item Metoda sloužící k uložení bodu do členské proměnné \textbf{points} určeného kliknutím myší nad kreslícím plátnem. Jejím návratovým typem je \textbf{void}.	
			\item Vstup
				\begin{itemize}
					\item \textbf{QMouseEvent *e}
				\end{itemize}
		\end{itemize}

	\item Metoda \textbf{setCH}
		\begin{itemize}
			\item Tato metoda slouží pro kontrolu duplicity generovaných bodů, pro kontrolu alespoň 3 bodů, k zavolání příslušného algoritmu pro vypočtení konvexní obálky a k omezení konvexní obálky na striktně konvexní obálku. Metoda počítá dobu trvání výpočetních algoritmů. Jejím návratovým typem je \textbf{double}.
			\item Vstup
				\begin{itemize}
					\item \textbf{std::string \&selected\_algorithm} - uživatelsky vybraný algoritmus pro počítání konvexní obálky
				\end{itemize}
			\item Výstup
				\begin{itemize}
					\item Čas trvání výpočtu.
				\end{itemize}
		\end{itemize}	

	\item Metoda \textbf{setRect}
		\begin{itemize}
			\item Tato metoda slouží pro zavolání algoritmu pro výpočte minimálního ohraničujícího obdélníku a jeho hlavní linie. Jejím návratovým typem je \textbf{void}.
			\item Vstup
				\begin{itemize}
					\item \textbf{bool draw\_dir\_line} - uživatelsky nastavený indikátor, zda-li se má vypočítat hlavní linie minimálního ohraničujícího obdélníku
				\end{itemize}
		\end{itemize}

	\item Metoda \textbf{setPoints}
		\begin{itemize}
			\item Metoda volající algoritmus pro generování bodů daného počtu a tvaru. Jejím návratovým typem je \textbf{void}.
			\item Vstup
				\begin{itemize}
					\item \textbf{QSizeF \&canvas\_size} - rozměr kreslícího plátna pro pozdější určení rozsahu generování bodů
					\item \textbf{int count} - počet bodů, jež se má generovat
					\item \textbf{std::string \&shape} - tvar, do kterého se body mají generovat
				\end{itemize}
		\end{itemize}

	\item Metoda \textbf{clearCanvas}
		\begin{itemize}
			\item Metoda, která maže obsah kreslícího okna. Jejím návratovým typem je \textbf{void}. Do metody nevstupují žádné parametry.
		\end{itemize}
\end{itemize}

\subsection{SortByXAsc, SortByYAsc, SortByAngleAsc}
Třídy sloužící jako sortovací kritérium - podle rostoucí souřadnice x resp. souřadnice y (při stejných souřadnicích x resp. y je druhým kritériem druhá souřadnice) a podle rostoucího úhlu mezi body (při stejném úhlu je druhým kritériem vzdálenosti mezi body).
\clearpage

\subsection{Widget}
Tato třída slouží ke komunikaci s GUI. Třída dědí od třídy QWidget. Všechny její metody slouží jako sloty k signálům z GUI, nemají žádné vstupní hodnoty a jejich návratovým typem je void. 

\begin{itemize}
	\item Metoda \textbf{on\_createCHButton\_clicked} - reaguje na zmáčknutí tlačítka pro vypočtení konvexní obálky, volá metodu \textbf{setCH} z třídy \textbf{Draw}, zapisuje čas výpočtu do GUI.

	\item Metoda \textbf{on\_generateButton\_clicked} - reaguje na zmáčknutí tlačítka pro generování bodů, volá metodu \textbf{setPoints} z třídy \textbf{Draw}.

	\item Metoda \textbf{on\_clearButton\_clicked} - reaguje na zmáčknutí tlačítka pro vymazání obsahu kreslícího plátna, volá metodu \textbf{clearCanvas} z třídy \textbf{Draw}.

	\item Metoda \textbf{on\_createRectButton\_clicked} - reaguje na zmáčknutí tlačítka pro vypočtení minimálního ohraničujícího obdélníka, volá metodu \textbf{setRect} z třídy \textbf{Draw}.

	\item Metoda \textbf{on\_helpButton\_clicked} - reaguje na zmáčknutí tlačítka pro volání nápovědy, volá okno s nápovědou \textbf{help\_dialog} z třídy \textbf{HelpDialog}.

\end{itemize} 

\vspace{3cm}
\subsection{HelpDialog}
Třída sloužící pro vykreslení okna s nápovědou.

\clearpage

\section{Přílohy}

\begin{itemize}
	\item Príloha č.1: Testování výpočetních dob algoritmů - "Testovani.pdf"
\end{itemize}
 %--------------------------------------------------------------------------------------------------------------------------------------------------------------------------
\clearpage
\section{Závěr}

\subsection{Politování se a vysvětlení našeho problému}

\subsection{Návrhy na vylepšení}


\clearpage
\section{Zdroje}

\begin{enumerate}
\item  BAYER, Tomáš. 2D triangulace, DMT [online][cit. 1.12.2018]. \\
Dostupné z: https://web.natur.cuni.cz/~bayertom/images/courses/Adk/adk5.pdf  \\

\item  BAYER, Tomáš. 2D triangulace, DMT [online][cit. 30.11.2018]. \\
Dostupné z: https://web.natur.cuni.cz/~bayertom/images/courses/Adk/adkcv3.pdf\\
%=======================================================================================
\end{enumerate}
\end{document}



 
